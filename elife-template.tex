%%%%%%%%%%%%%%%%%%%%%%%%%%%%%%%%%%%%%%%%%%%%%%%%%%%%%%%%%%%%
%%% ELIFE ARTICLE TEMPLATE
%%%%%%%%%%%%%%%%%%%%%%%%%%%%%%%%%%%%%%%%%%%%%%%%%%%%%%%%%%%%
%%% PREAMBLE 
\documentclass[9pt,lineno]{elife}

\usepackage[version=4]{mhchem}
\usepackage{siunitx}
\usepackage{makecell}
\usepackage{mhchem}
\usepackage[colorinlistoftodos]{todonotes} % To add todos.
\DeclareSIUnit\Molar{M}

%%%%%%%%%%%%%%%%%%%%%%%%%%%%%%%%%%%%%%%%%%%%%%%%%%%%%%%%%%%%
%%% ARTICLE SETUP
%%%%%%%%%%%%%%%%%%%%%%%%%%%%%%%%%%%%%%%%%%%%%%%%%%%%%%%%%%%%
\title{Binding thermodynamics of host-guest systems with SMIRNOFF99Frosst from the Open Force Field Group}

\author[1]{David R. Slochower}
\author[2]{Niel M. Henriksen}
% \author[3]{Michael R. Shirts}
\author[5]{John D. Chodera}
% \author[4]{David L. Mobley}
\author[1]{Michael K. Gilson}

\affil[1]{Skaggs School of Pharmacy and Pharmaceutical Sciences, University of California, San Diego, La Jolla, CA 92093, USA}
\affil[2]{Atomwise, Inc., San Francisco, CA 94105, USA}
% \affil[3]{Department of Chemical and Biological Engineering, University of Colorado Boulder, Boulder, CO 80309}
% \affil[4]{Department of Pharmaceutical Sciences and Department of Chemistry, University of California, Irvine, CA 92697, USA}
\affil[5]{Computational and Systems Biology Program, Sloan Kettering Institute, Memorial Sloan Kettering Cancer Center, New York, NY 10065}

\corr{mgilson@ucsd.edu}{MKG}

\newif\ifdraft
\drafttrue
\ifdraft
 \newcommand{\drsnote}[1]{ {\textcolor{red} { [DRS: #1] }}}
\else
 \newcommand{\drsnote}[1]{}
\fi


%%%%%%%%%%%%%%%%%%%%%%%%%%%%%%%%%%%%%%%%%%%%%%%%%%%%%%%%%%%%
%%% ARTICLE START
%%%%%%%%%%%%%%%%%%%%%%%%%%%%%%%%%%%%%%%%%%%%%%%%%%%%%%%%%%%%

\begin{document}

\maketitle
\drsnote{I have only included authors who reviewed the outline so far.}

\begin{abstract}

\end{abstract}

\section{Introduction}


\section{Theory and motivation}

\section{Methods}
\subsection{Choice of host-guest systems}
In this study, we report the binding thermodynamics of 43 host-guest complexes (Figure \ref{fig:host-guest-pairs}) computed using three different force fields. 
The complexes consist of either $\alpha$- or $\beta$-cyclodextrin as host molecule and a series of ammonium, carboxylate, or cyclic alcohol small molecule guests.
Cyclodextrins are cyclic polymers consisting of six ($\alpha$CD) or seven ($\beta$CD) glucose monomers in the shape of a truncated cone.
The equilibrium constants and standard molar enthalpies of binding for these 43 complexes have been measured using isothermal titration calorimetry \cite{rekharsky_thermodynamic_1997} and computationally in \cite{henriksen_evaluating_2017}.
As in \cite{henriksen_evaluating_2017}, only a single structural isomer was considered for the 1-methylammonium guests.
\drsnote{Niel described this as a stereoisomer, but I think it is actually a structural isomer (cf.~1-butylamine and tert-butylamine), unless I'm misunderstanding.}

\begin{figure}[tb]
\centering
\includegraphics[width=\textwidth]{images/host-guest-pairs.png}
\caption{Structures of the two cyclodextrin hosts and 33 guest molecules in this study which together comprise 43 unique host-guest pairs. \drsnote{This figure is taken from Niel's paper. If we don't modify it, we should get permission for its reuse.}}
\label{fig:host-guest-pairs}
\end{figure}

\subsection{Application of force field parameters}

% The simulation system for each host-guest pair consisted of: 2000 and 2210 water molecules for αCD and βCD hosts, respectively; appropriate Na+ or Cl- ions to neutralize the total system; additional 25 mM NaCl or Na¬H2PO4/Na2HPO4 buffer to match experimental conditions37; and a single CD host and guest molecule. As part of this investigation, we examined the influence of water model, ion parameters, partial charge assignment methods, and CD force field. Four water models were studied: TIP3P,18 TIP4Pew,19 SPC/E,20 and OPC.21 Ion parameters for Na+ and Cl- were taken from Joung and Cheatham's work,43 in which the parameters were tuned for each water model, except for simulations with the OPC water model, for which we tried both Joung and Cheatham's TIP4Pew ion parameters as well as parameters tuned specifically for matching hydration energies in OPC (distributed with AMBER; see Section 3.5 and 3.6 in the AMBER 16 manual). Results with Na¬H2PO4/Na2HPO4 buffer are reported only for the OPC water model, as these ions were observed to aggregate unrealistically into ordered structures when used with other water models. The parameters for these phosphate ions were based on those suggested by Steinbrecher et al.44 The CD host molecules were parameterized with either the Q4MD-CD force field,45 which uses RESP derived charges and draws many Lennard-Jones and valence parameters from GLYCAM04,46–48 or a crude "minimal-effort" version we created as a control. In the latter approach, a single glucose molecule, with methoxy caps on the O1 and O4 alcohols, was parameterized with AM1-BCC49,50 partial charges and GAFF51 Lennard-Jones and valence parameters, using AMBER's antechamber and parmchk2 utilites. Guest molecules used either RESP partial charges computed with the R.E.D. Server52 or AM1-BCC charges computed with antechamber. Guest Lennard-Jones and valence parameters were obtained from GAFF. A complete list of the force field combinations tested is provided in Table 2.

We sought to compare force fields directly, and as such, attempted to minimize additional differences between the simulation sets.
In all simulations, we applied AM1-BCC \cite{jakalian_fast_2000, jakalian_fast_2002} partial atomic charges to both the host and guest molecules using the \texttt{antechamber} program.
The host charges were calculated using a single glucose molecule with methoxy caps on the O1 and O4 alcohols (Figure \ref{fig:atom-names}); each glucose monomer in the cyclodextrin polymer has identical charges.
We used TIP3P water \cite{jorgensen_comparison_1983} and Joung-Cheatham monovalent ion parameters for \ce{Na+} and \ce{Cl-} \cite{joung_determination_2008} in each simulation set.

GAFF v1.7 bond, angle, torsion, and Lennard-Jones parameters were applied using the \texttt{tleap} program distributed with AmberTools16. These simulations were performed as part of Henriksen, et al.~\cite{henriksen_evaluating_2017} and are described in additional detail therein.

GAFF v2.1 parameters were applied in an identical manner to the GAFF v1.7 parameters, using the \texttt{tleap} program distributed with AmberTools18 and substituting \texttt{leaprc.gaff} for \texttt{leaprc.gaff2} in the \texttt{tleap} input file.
In GAFF v2.1, the bond and angle parameters have been updated to reproduce small molecule geometries obtained from high-level quantum mechanical calculations.
The force constants for the bond and angle parameters were tuned to reproduce the vibrational spectra of over 600 molecules.
The torsion parameters were optimized to reproduce the rotational potential energy surface of 400 model compounds.
Finally, the Lennard-Jones coefficients were redeveloped to reproduce interaction energies and pure liquid properties.
\drsnote{We might want to cite \texttt{gaff2.dat} somehow, because this is where I obtained the above information; we could also cite personal communications with Junmei Wang based on our conversations at the Amber developers' meeting.}

To apply SMIRNOFF99Frosst parameters, we followed a \href{https://github.com/slochower/smirnoff-host-guest/blob/master/02-convert-APR-files.ipynb}{multistep process}, beginning with the prepared GAFF v1.7 files.
The host and guest molecules were parameterized with the Open Force Field Toolkit version 0.0.3, SMIRNOFF version 1.0, and SMIRNOFF99Frosst version 1.0.5.
Once parameterized with SMIRNOFF99Frosst, ParmEd \cite{shirts_lessons_2017} was used to combine the host and guest with the solvent and ions, which retained their parameters as TIP3P water and Joung-Cheatham ion parameters, respectively.
\drsnote{I decided to skip a detailed listing of what I did, and instead link to the Jupyter notebook I used, because these instructions may change with future toolkit versions, and in fact, it is already out of date in some ways. If this is a good way to cite this, I can upload that GitHub repository to Zenodo, get a DOI, and cite the DOI. It may also be feasible to include this as SI, but that can get messy for anything other than trivial code listings.}
% In each window of each calculation:
% \begin{enumerate}
%     \item The host and guest coordinates are split from the water and ions.
%     \item The OpenEye toolkit is used to parse the host and guest coordinates and generate a topology.
%     \item The list of OpenEye \texttt{OEMols}, along with the molecular topology, is passed to the 
% \end{enumerate}

\subsection{Thermodynamic calculations}

We used the attach-pull-release (APR) method as implemented in the open source package pAPRika version 0.0.3, to calculate absolute binding free energies.
A complete description of the APR method has been characterized in the literature \cite{henriksen_computational_2015, velez-vega_overcoming_2013, muddana_sampl4_2014, fenley_bridging_2014}.

The attachment and release phases consisted of 15 independent windows; the pulling phase consisted of 45 independent windows.
During the attachment phase, the force constants on the host and guest are scaled by a $\lambda$ parameter that goes from $\lambda = 0$, when all restraints are turned off, to $\lambda = 1$, when all restraints reach their maximum force constant. The $\lambda$ windows are more densely spaced where the force constant is smaller to improve sampling along highly curved regions of the potential of mean force.
Conformational restraints were applied between neighboring glucose units of the cyclodextrin to limit the incursion of monomers into the host cavity. These restraints were applied along the pseudodihedrals \ce{{O5}_n-{C1}_n-{O1}_n-{C4}_{n+1}} and \ce{{C1}_n-{O1}_n-{C4}_{n+1}-{C5}_{n+1}} to improve convergence and sampling of the bound state (see Figure \ref{fig:atom-names} for atom naming conventions). To further improve convergence of the bound state, we applied a hard wall restraint that confined the guest molecule to within a sphere of 12.3 and 13.5 \AA\ of $\alpha$CD and $\beta$CD, respectively. 

During the pulling phase, the $\lambda$ parameter represented a distance restraint and is increased uniformly in 45 increments of 0.4 \AA\, yielding windows that separate the host and guest by 18 \AA. An explicit release calculation is performed to turn off the conformational restraints on the host in the absence of guest molecules. This is performed once for each cyclodextrin. An analytic release calculation is performed to compute the work of moving the guest from the restricted volume enforced by the APR restraints to standard state at 1 M concentration. \drsnote{Needs editing.}

Due to the lack of symmetry of cyclodextrin and the small molecule guests, there are generally two distinct binding poses. We separately compute the binding free energy and enthalpy for each orientation \cite{henriksen_computational_2015} and combine the results to produce a single value for each host-guest combination.
% For simple guest molecules, with one polar functional group, there are generally two possible binding modes in the CD cavity: one with the polar group oriented out of the CD opening with primary alcohols, and one with the polar group oriented out of the wider opening with secondary alcohols (Figure 2). However, experimental data report binding free energy and enthalpy values based on a Boltzmann-weighted ensemble of these two orientations. In order to compare with experiment, we separately compute and appropriately combine the binding free energy and enthalpy for each orientation, as previously described.40 

Thermodynamic integration was used to compute the binding free energy. The binding enthalpy was computed as the difference in mean potential energy of the bound state (in the absence of any restraints) and the unbound state (where the guest is held far away from the host, but the conformational restraints on the host are disabled).
% The binding enthalpy was computed by subtracting the average potential energy of the last window of the release phase, in which the guest has been pulled far away from the host, from the mean energy of the first window of the attachment phase, in which restraint force constants are off. The uncertainties for average properties calculated in each window (e.g., restraint coordinate values and potential energies) were estimated with the blocking method42 and propagated into the final reported values using bootstrapping for the thermodynamic integration calculations, or addition in quadrature for the binding enthalpy calculations.
Uncertanties were computed using blocking analysis \cite{flyvbjerg_error_1989}.

\subsection{Simulations}
Simulations were performed with the \texttt{pmemd.cuda} module of AMBER 16 (calculations of the GAFF v1.7 force field) and AMBER 18 (calculations of the GAFF v2.1 and SMIRNOFF99Frosst force fields) molecular dynamics software. Each window for each system was independently solvated and simulated.

Solvation consisted of 2000 TIP3P waters for the $\alpha$CD systems and 2210 waters for the $\beta$CD systems in an orthorhomic box. Each simulation contained enough \ce{Na+} or \ce{Cl-} ions to neutralize the host-guest complex  and an additional 50 mM \ce{NaCl} to match the experimental conditions in \cite{rekharsky_thermodynamic_1997}. The host and guest were oriented via non-interacting dummy atoms along the simulation box $z$ axis to minimize the amount of solvent required. Hydrogen mass repartitioning was used to adjust the mass of hydrogen atoms to 3.024 Da by transferring mass from bonded heavy atoms, enabling a simulation timestep of 4 fs. Equilibration consisted of 50,000 steps of energy minimization, 100 ps of heating from 0 to 300 K, followed by 2000 ps of additional NPT simulations. A Langevin thermostat, the Monte Carlo barostat, a nonbonded cutoff of 9 \AA\ and default PME parameters, were used for the NPT simulations.

Production NPT simulations were run for a minimum of 5 ns and maximum of 50 ns per window, except for the endpoint windows used to calculate the enthalpy, which were simulated for 1 microsecond. In the GAFF v1.7 and GAFF v2.1 simulations, the exact length of each window's simulation was determined by the restraint coordinate uncertainty, estimated via blocking analysis \cite{flyvbjerg_error_1989}. That is, for restraint energy $U$ in window $i$, each window was simulated until the variable
\begin{equation}
    w(\lambda_i) = 
    \begin{cases}
    \left\langle \frac{\partial U}{\partial \lambda_i} \right\rangle_\text{SEM} \cdot \frac{\lambda_{i+1}}{2} & i = 0 \\
    \left\langle \frac{\partial U}{\partial \lambda_i} \right\rangle_\text{SEM} \cdot \frac{\lambda_{i+1} - \lambda_{i-1}}{2} & i \in [1, N-1] \\
    \left\langle \frac{\partial U}{\partial \lambda_i} \right\rangle_\text{SEM} \cdot \frac{1 - \lambda_{i-1}}{2} & i = N \\
    \end{cases}
\end{equation}
was below a threshold = 0.02 kcal/mol during the attach phase and 0.1 kcal/mol during the pull phase.

\drsnote{This is ambiguous because $i=0$ and $i=N$ during attach and release are run for longer than this.}

SMIRNOFF99Frosst simulations calculations were run for 10 ns per window.

%All simulations were performed with either the AMBER 14 or 163 molecular dynamics software. For each simulation window, the entire system was rebuilt using AMBER's tleap utility, thus ensuring that the initial solvent configuration in each window was independent of that in the other windows. The periodic box was orthorhombic, with dimensions of approximately 35x35x51 Å and 37x37x51 Å for αCD and βCD simulations, respectively. The host-guest system was oriented in the simulation box, via non-interacting anchor particles, which allowed the APR process to occur along the long (z) axis of the box; see Henriksen et al. for details.40  The parameter/topology file was then hydrogen mass repartitioned (HMR) with AMBER's parmed utility, a process which increases solute hydrogen atom masses to 3.024 Da by transferring mass from adjacent heavy atoms, thereby allowing larger simulation time steps without integration errors or significant changes to the thermodynamics.40,53,54 All equilibration and production simulations were carried out with the GPU-capable pmemd.cuda MD engine. Equilibration consisted of 50,000 steps of energy minimization, followed by 100 ps of NVT warming from 0 - 298.15 K, and 2000 ps of NPT equilibration at 298.15 K.  Production simulations were run for a minimum of 5.0 ns out to a maximum of 50 ns, with the exact length determined by a threshold in the restraint coordinate uncertainty. All simulations used a time step of 4 fs, enabled by HMR, with a Langevin thermostat55 and a Monte-Carlo barostat.56 The non-bonded cutoff was set to 9.0 Å and the default AMBER PME parameters (identical for AMBER 14 and 16) were employed.


\section{Results and discussion}
SMIRNOFF99Frosst does as well as GAFF, despite far fewer numerical force field parameters.

\subsection{Binding free energies and enthalpies}

\begin{figure}[tb]
\centering
\includegraphics[width=0.49\textwidth]{images/SMIRNOFF99Frosst-vs-Experiment-dG.pdf}
\includegraphics[width=0.49\textwidth]{images/SMIRNOFF99Frosst-vs-Experiment-dH.pdf}
\caption{Caption}
\label{fig:S99-vs-experiment}
\end{figure}

\begin{figure}[tb]
\centering
\includegraphics[width=0.49\textwidth]{{images/GAFF-v1.7-vs-Experiment-dG}.pdf}
\includegraphics[width=0.49\textwidth]{{images/GAFF-v1.7-vs-Experiment-dH}.pdf}
\caption{Caption}
\label{fig:GAFF-v17-vs-experiment}
\end{figure}

\begin{figure}[tb]
\centering
\includegraphics[width=0.49\textwidth]{{images/GAFF-v2.1-vs-Experiment-dG}.pdf}
\includegraphics[width=0.49\textwidth]{{images/GAFF-v2.1-vs-Experiment-dH}.pdf}
\caption{Caption}
\label{fig:GAFF-v21-vs-experiment}
\end{figure}


\begin{figure}[tb]
\centering
\includegraphics[width=0.49\textwidth]{{images/SMIRNOFF99Frosst-vs-GAFF-v1.7-bond-k}.pdf}
\includegraphics[width=0.49\textwidth]{{images/SMIRNOFF99Frosst-vs-GAFF-v2.1-bond-k}.pdf}
\includegraphics[width=0.49\textwidth]{{images/SMIRNOFF99Frosst-vs-GAFF-v1.7-bond-req}.pdf}
\includegraphics[width=0.49\textwidth]{{images/SMIRNOFF99Frosst-vs-GAFF-v2.1-bond-req}.pdf}
\caption{Caption}
\label{fig:bond}
\end{figure}

\begin{figure}[tb]
\centering
\includegraphics[width=0.49\textwidth]{{images/SMIRNOFF99Frosst-vs-GAFF-v1.7-ang-k}.pdf}
\includegraphics[width=0.49\textwidth]{{images/SMIRNOFF99Frosst-vs-GAFF-v2.1-ang-k}.pdf}
\includegraphics[width=0.49\textwidth]{{images/SMIRNOFF99Frosst-vs-GAFF-v1.7-ang-theta}.pdf}
\includegraphics[width=0.49\textwidth]{{images/SMIRNOFF99Frosst-vs-GAFF-v2.1-ang-theta}.pdf}
\caption{Caption}
\label{fig:angle}
\end{figure}


\drsnote{I wonder whether I should rename "SMIRNOFF99Frosst" to whatever the new versioning string actually is.}

Both SMIRNOFF99Frosst and GAFF v1.7 systematically underestimate the binding free energy for cyclic alcohols except for beta-cyclodextrin with cyclopentanol (b-cpe).

GAFF v1.7 has better correlations wiht the experimental data, specifcally with the orange points (ammonium, I think), which gets even better in GAFF v2.1, despite an increased systematic offset. 


\susbsection{Parameter comparisons with GAFF v1.7 and GAFF v2.1}

\begin{figure}[tb]
\centering
\includegraphics[width=0.49\textwidth]{{images/SMIRNOFF99Frosst-vs-GAFF-v1.7-sigma}.pdf}
\includegraphics[width=0.49\textwidth]{{images/SMIRNOFF99Frosst-vs-GAFF-v2.1-sigma}.pdf}
\caption{Caption}
\label{fig:sigma}
\end{figure}


\begin{figure}[tb]
\centering
\includegraphics[width=0.49\textwidth]{{images/SMIRNOFF99Frosst-vs-GAFF-v1.7-epsilon}.pdf}
\includegraphics[width=0.49\textwidth]{{images/SMIRNOFF99Frosst-vs-GAFF-v2.1-epsilon}.pdf}
\caption{Caption}
\label{fig:epsilon}
\end{figure}

% About Q4MD: . This force field is a combination of GLYCAM04(44-46) and Amber99SB(47) and has been reported to have good performances toward CDs using the Amber 10 software(48) in a recent paper.(43)
% https://pubs.acs.org/doi/full/10.1021/jp308416p

% Q4MD citation: https://pubs.rsc.org/en/content/articlelanding/2011/cp/c1cp20854c/unauth#!divAbstract

\subsection{Guest preferences between binding orientations}


Cyclic ones can’t push away form theydorxyls so they prefer secondary (especially for alpha).

The amines stick out into solution.


\begin{figure}[tb]
\centering
\includegraphics[width=0.49\textwidth]{{images/SMIRNOFF99Frosst-vs-GAFF-v1.7-dG-by-orientation-highlight}.pdf}
\includegraphics[width=0.49\textwidth]{{images/SMIRNOFF99Frosst-vs-GAFF-v2.1-dG-by-orientation-highlight}.pdf}
\caption{Caption}
\label{fig:epsilon}
\end{figure}

\begin{figure}[tb]
\centering
\includegraphics[width=0.49\textwidth]{{images/SMIRNOFF99Frosst-dG-by-orientation-highlight}.pdf}
\includegraphics[width=0.49\textwidth]{{images/GAFF-v1.7-dG-by-orientation-highlight}.pdf}
\includegraphics[width=0.49\textwidth]{{images/GAFF-v2.1-dG-by-orientation-highlight}.pdf}

\caption{Caption \drsnote{I think I've done the P/S labeling backwards.}}
\label{fig:epsilon}
\end{figure}


\subsection{Guest preferences between $\alpha$- and $\beta$-cyclodextrin}

For beta, not as big a difference between orientations for the cyclic ones. Until you get to cyclo-octanol, whih is not due to clashing, genuine steric repuslion.

A-coc-p doesn’t bind in teh cavity. Secondayr is closer, but not greawt either.

\begin{figure}[tb]
\centering
\includegraphics[width=0.49\textwidth]{{images/SMIRNOFF99Frosst-vs-Experiment-dG-by-cyclodextrin}.pdf}
\caption{Caption}
\label{fig:epsilon}
\end{figure}

\subsection{Functional group patterns}

\begin{figure}[tb]
\centering
\includegraphics[width=0.49\textwidth]{{images/SMIRNOFF99Frosst-vs-Experiment-dG-highlight-alpha-alcohols}.pdf}
\includegraphics[width=0.49\textwidth]{{images/SMIRNOFF99Frosst-vs-Experiment-dG-highlight-beta-alcohols}.pdf}
\includegraphics[width=0.49\textwidth]{{images/SMIRNOFF99Frosst-vs-Experiment-dG-highlight-alpha-ammonium}.pdf}
\includegraphics[width=0.49\textwidth]{{images/SMIRNOFF99Frosst-vs-Experiment-dG-highlight-beta-ammonium}.pdf}
\includegraphics[width=0.49\textwidth]{{images/SMIRNOFF99Frosst-vs-Experiment-dG-highlight-alpha-carboxylates}.pdf}
\includegraphics[width=0.49\textwidth]{{images/SMIRNOFF99Frosst-vs-Experiment-dG-highlight-beta-carboxylates}.pdf}
\caption{Caption}
\label{fig:epsilon}
\end{figure}


\subsection{Structural insights}

The flexibility of SMIRNOFF99Frosst is similiar to that of GAFF 1.7.
The flexibility of GAFF v2.1 is similar to other carbohydrate specific force fields.

Include the torsion profile plots, too.

\begin{figure}[tb]
\centering
\includegraphics[width=0.49\textwidth]{{images/SMIRNOFF99Frosst-vs-GAFF-v1.7-H1-C2-C2-O2}.pdf}
\includegraphics[width=0.49\textwidth]{{images/SMIRNOFF99Frosst-vs-GAFF-v2.1-C1-O1-C4-C3}.pdf}
\includegraphics[width=0.49\textwidth]{{images/SMIRNOFF99Frosst-vs-GAFF-v2.1-O1-C4-C3-O3}.pdf}
\includegraphics[width=0.49\textwidth]{{images/SMIRNOFF99Frosst-vs-GAFF-v2.1-O1-C4-C5-O5}.pdf}
\caption{Caption}
\label{fig:dihedral-energy}
\end{figure}

% See SI table 1: https://pubs.acs.org/doi/full/10.1021/jp308416p#notes-1

\begin{figure}[tb]
\centering
\includegraphics[width=0.49\textwidth]{{images/cyclodextrin-flexibility-comparison}.png}
\caption{Caption}
\label{fig:cyclodextrin-flexbility}
\end{figure}

\section{Conclusions and future work}

%%%%%%%%%%%%%%%%%%%%%%%%%%%%%%%%%%%%%%%%%%%%%%%%%%%%%%%%%%%%%%%%%%%%%%%%%%%%%%%%%%%%%%%%%%%%%%%%%%%%%%
% Code and Data Availability
%%%%%%%%%%%%%%%%%%%%%%%%%%%%%%%%%%%%%%%%%%%%%%%%%%%%%%%%%%%%%%%%%%%%%%%%%%%%%%%%%%%%%%%%%%%%%%%%%%%%%%

\section{Code and data availability}
\begin{itemize}
	\item Code to convert from GAFF to SMIRNOFF99Frosst
	\item Code to setup the simulations using pAPRika
	\item Code to analyze the simulation results
	\item OpenFF code and link to SMIRNOFF99Frosst
\end{itemize}
%%%%%%%%%%%%%%%%%%%%%%%%%%%%%%%%%%%%%%%%%%%%%%%%%%%%%%%%%%%%%%%%%%%%%%%%%%%%%%%%%%%%%%%%%%%%%%%%%%%%%%
% Author Contributions 
%%%%%%%%%%%%%%%%%%%%%%%%%%%%%%%%%%%%%%%%%%%%%%%%%%%%%%%%%%%%%%%%%%%%%%%%%%%%%%%%%%%%%%%%%%%%%%%%%%%%%%
\section{Author Contributions}
Conceptualization, DRS, NMH, JDC, MKG; Methodology, DRS, NMH; Software, DRS, NMH; Formal Analysis, DRS, NMH, JDC, MKG; Investigation, DRS, NMH; Resources, MKG, JDC;  Data Curation, DRS, NMH; Writing-Original Draft, DRS, NMH; Writing - Review and Editing, DRS, NMH, JDC, MKG; Visualization, DRS; Supervision, JDC, MKG; Project Administration, MKG; Funding Acquisition, MKG.

%%%%%%%%%%%%%%%%%%%%%%%%%%%%%%%%%%%%%%%%%%%%%%%%%%%%%%%%%%%%%%%%%%%%%%%%%%%%%%%%%%%%%%%%%%%%%%%%%%%%%%
% Acknowledgments 
%%%%%%%%%%%%%%%%%%%%%%%%%%%%%%%%%%%%%%%%%%%%%%%%%%%%%%%%%%%%%%%%%%%%%%%%%%%%%%%%%%%%%%%%%%%%%%%%%%%%%%
\section{Acknowledgments}
%%%%%%%%%%%%%%%%%%%%%%%%%%%%%%%%%%%%%%%%%%%%%%%%%%%%%%%%%%%%%%%%%%%%%%%%%%%%%%%%%%%%%%%%%%%%%%%%%%%%%%
% Disclosures 
%%%%%%%%%%%%%%%%%%%%%%%%%%%%%%%%%%%%%%%%%%%%%%%%%%%%%%%%%%%%%%%%%%%%%%%%%%%%%%%%%%%%%%%%%%%%%%%%%%%%%%
\section{Disclosures}
The authors declare the following competing financial interest(s): MKG has an equity interest in and is a cofounder and scientific advisor of VeraChem LLC.

\bibliography{references}

%%%%%%%%%%%%%%%%%%%%%%%%%%%%%%%%%%%%%%%%%%%%%%%%%%%%%%%%%%%%
%%% APPENDICES
%%%%%%%%%%%%%%%%%%%%%%%%%%%%%%%%%%%%%%%%%%%%%%%%%%%%%%%%%%%%

\appendix
\begin{figure}[tb]
\centering
\includegraphics[width=0.49\textwidth]{images/atom-names-trimer.png}
\includegraphics[width=0.49\textwidth]{images/gaff-atom-types.png}
\caption{Atom names (left) and GAFF atom types (right) for a glucose monomer in alpha-cyclodextrin, shown with two flanking monomers. The remaining three glucose monomers are hidden for clarity.}
\label{fig:atom-names}
\end{figure}

\begin{figure}[tb]
\centering
\includegraphics[width=0.49\textwidth]{{images/SMIRNOFF99Frosst-dG-by-orientation-all}.pdf}
\includegraphics[width=0.49\textwidth]{{images/GAFF-v1.7-dG-by-orientation-all}.pdf}
\includegraphics[width=0.49\textwidth]{{images/GAFF-v2.1-dG-by-orientation-all}.pdf}

\caption{Caption.}
\label{fig:epsilon}
\end{figure}


\section{List of abbreviations}
APR, attach-pull-release; CD, cyclodextrin; GAFF, Generalized AMBER Force Field
\end{document}

%%%%%%%%%%%%%%%%%%%%%%%%%%%%%%%%%%%%%%%%%%%%%%%%%%%%%%%%%%%%
%%% ELIFE ARTICLE TEMPLATE
%%%%%%%%%%%%%%%%%%%%%%%%%%%%%%%%%%%%%%%%%%%%%%%%%%%%%%%%%%%%
%%% PREAMBLE 
\documentclass[9pt,lineno]{elife}

\usepackage[version=4]{mhchem}
\usepackage{siunitx}
\usepackage{makecell}
\usepackage[colorinlistoftodos]{todonotes} % To add todos.
\DeclareSIUnit\Molar{M}

%%%%%%%%%%%%%%%%%%%%%%%%%%%%%%%%%%%%%%%%%%%%%%%%%%%%%%%%%%%%
%%% ARTICLE SETUP
%%%%%%%%%%%%%%%%%%%%%%%%%%%%%%%%%%%%%%%%%%%%%%%%%%%%%%%%%%%%
\title{Binding thermodynamics of host-guest systems with SMIRNOFF99Frosst from the Open Force Field Group}

\author[1]{David R. Slochower}
\author[2]{Niel M. Henriksen}
% \author[3]{Michael R. Shirts}
\author[5]{John D. Chodera}
% \author[4]{David L. Mobley}
\author[1]{Michael K. Gilson}

\affil[1]{Skaggs School of Pharmacy and Pharmaceutical Sciences, University of California, San Diego, La Jolla, CA 92093, USA}
\affil[2]{Atomwise, Inc., San Francisco, CA 94105, USA}
% \affil[3]{Department of Chemical and Biological Engineering, University of Colorado Boulder, Boulder, CO 80309}
% \affil[4]{Department of Pharmaceutical Sciences and Department of Chemistry, University of California, Irvine, CA 92697, USA}
\affil[5]{Computational and Systems Biology Program, Sloan Kettering Institute, Memorial Sloan Kettering Cancer Center, New York, NY 10065}

\corr{mgilson@ucsd.edu}{MKG}

\newif\ifdraft
\drafttrue
\ifdraft
 \newcommand{\drsnote}[1]{ {\textcolor{red} { [DRS: #1] }}}
\else
 \newcommand{\drsnote}[1]{}
\fi


%%%%%%%%%%%%%%%%%%%%%%%%%%%%%%%%%%%%%%%%%%%%%%%%%%%%%%%%%%%%
%%% ARTICLE START
%%%%%%%%%%%%%%%%%%%%%%%%%%%%%%%%%%%%%%%%%%%%%%%%%%%%%%%%%%%%

\begin{document}

\maketitle
\drsnote{I have only included authors who reviewed the outline so far.}

\begin{abstract}

\end{abstract}

\section{Introduction}


\section{Theory and motivation}

\section{Methods}
\subsection{Choice of host-guest systems}
In this study, we report the binding thermodynamics of 43 host-guest complexes (Figure \ref{fig:host-guest-pairs}) computed using three different force fields. 
The complexes consist of either $\alpha$- or $\beta$-cyclodextrin as host molecule and a series of ammonium, carboxylate, or cyclic alcohol small molecule guests.
Cyclodextrins are cyclic polymers consisting of six ($\alpha$) or seven ($\beta$) glucose monomers in the shape of a truncated cone.
The equilibrium constants and standard molar enthalpies of binding for these 43 complexes have been measured using isothermal titration calorimetry \cite{rekharsky_thermodynamic_1997} and computationally in \cite{henriksen_evaluating_2017}.
As in \cite{henriksen_evaluating_2017}, only a single structural isomer was considered for the 1-methylammonium guests.
\drsnote{Niel described this as a stereoisomer, but I think it is actually a structural isomer, unless I'm misunderstanding.}

\begin{figure}[tb]
\centering
\includegraphics[width=\textwidth]{images/host-guest-pairs.png}
\caption{Structures of the two cyclodextrin hosts and 33 guest molecules in this study which together comprise 43 unique host-guest pairs. \drsnote{This figure is taken from Niel's paper. If we don't modify it, we should get permission for its reuse.}}
\label{fig:host-guest-pairs}
\end{figure}

\subsection{Application of force field parameters}
We sought to compare force fields directly, and as such, attempted to minimize additional differences between the simulations.
In all simulations, we applied AM1-BCC \cite{jakalian_fast_2000, jakalian_fast_2002} partial atomic charges to both the host and guest molecules using the \texttt{antechamber} program.
The host charges were calculated using a single glucose molecule with methoxy caps on the O1 and O4 alcohols (Figure \ref{fig:atom-names}); each glucose monomer in the cyclodextrin polymer has identical charges.

GAFF v1.7 bond, angle, torsion, and Lennard-Jones parameters were applied using the \texttt{tleap} program distributed with AmberTools16.
These simulations were performed as part of Henriksen, et al.~\cite{henriksen_evaluating_2017}.

GAFF v2.1 parameters were applied in an identical manner to the GAFF v1.7 parameters, using the \texttt{tleap} program distributed with AmberTools18 and the \texttt{leaprc.gaff2} file.
In GAFF v2.1, the bond and angle parameters have been updated to reproduce small molecule geometries obtained from high-level quantum mechanical calculations.
The force constants for the bond and angle parameters were tuned to reproduce the vibrational spectra of over 600 molecules.
The torsion parameters were optimized to reproduce the rotational potential energy surface of 400 model compounds.
Finally, the Lennard-Jones coefficients were redeveloped to reproduce interaction energies and pure liquid properties.

To apply SMIRNOFF99Frosst, we followed a \href{https://github.com/slochower/smirnoff-host-guest/blob/master/02-convert-APR-files.ipynb}{multistep process}, beginning with the prepared GAFF v1.7 files.
First, we split the structure into the host-guest complex and the solvent.
From the host-guest complex, we created a topology and Tripos mol2-file.
We used OpenEye tools to turn this into a topology.
Parameterized using the OpenFF toolkit.
Used ParmEd (http://dx.doi.org/10.1007/s10822-016-9977-1) to combine the host and guest with teh solvent.

In this way, we replace the force field without changing the molecular coordinates.


These calculations were performed with the Open Force Field Toolkit version 0.0.3, SMIRNOFF version 1.0, and SMIRNOFF99Frosst version 1.0.5.


\subsection{Simulations}
GAFF v1.7 simulations were performed with AMBER16; GAFF v2.1 and SMIRNOFF99Frosst simulations were performed with AMBER18 molecular dynamics software.

- 2000 or 2210 waters.
- TIP3P, Na+/Cl- only



Molecular coordinates for the 43 host-guest complexes parameterized with the GAFF v1.7 force field, AM1-BCC partial atomic charges on both host and guest molecules, and solvated with the TIP3P water model were used from our previous study \cite{henriksen_evaluating_2017}.
Each window in each system was then 


\subsection{Thermodynamic calculations}
We used the attach-pull-release (APR) method as implemented in the open source package pAPRika, version 0.0.3.

A complete description of the APR method is described in \cite{henriksen_computational_2015}.

\section{Results and discussion}
SMIRNOFF99Frosst does as well as GAFF, despite far fewer numerical force field parameters.

\subsection{Parameter differences}

\subsection{Binding free energies and enthalpies}

\begin{figure}[tb]
\centering
\includegraphics[width=0.49\textwidth]{images/SMIRNOFF99Frosst-vs-Experiment-dG.pdf}
\includegraphics[width=0.49\textwidth]{images/SMIRNOFF99Frosst-vs-Experiment-dH.pdf}
\caption{Caption}
\label{fig:S99-vs-experiment}
\end{figure}

\begin{figure}[tb]
\centering
\includegraphics[width=0.49\textwidth]{{images/GAFF-v1.7-vs-Experiment-dG}.pdf}
\includegraphics[width=0.49\textwidth]{{images/GAFF-v1.7-vs-Experiment-dH}.pdf}
\caption{Caption}
\label{fig:GAFF-v17-vs-experiment}
\end{figure}

\begin{figure}[tb]
\centering
\includegraphics[width=0.49\textwidth]{{images/GAFF-v2.1-vs-Experiment-dG}.pdf}
\includegraphics[width=0.49\textwidth]{{images/GAFF-v2.1-vs-Experiment-dH}.pdf}
\caption{Caption}
\label{fig:GAFF-v21-vs-experiment}
\end{figure}

Both SMIRNOFF99Frosst and GAFF v1.7 systematically underestimate the binding free energy for cyclic alcohols except for beta-cyclodextrin with cyclopentanol (b-cpe).

GAFF v1.7 has better correlations wiht the experimental data, specifcally with the orange points (ammonium, I think), which gets even better in GAFF v2.1, despite an increased systematic offset. 


\subsection{Guest preferences between binding orientations}


Cyclic ones can’t push away form theydorxyls so they prefer secondary (especially for alpha).

The amines stick out into solution.


\subsection{Guest preferences between $\alpha$- and $\beta$-cyclodextrin}

For beta, not as big a difference between orientations for the cyclic ones. Until you get to cyclo-octanol, whih is not due to clashing, genuine steric repuslion.

A-coc-p doesn’t bind in teh cavity. Secondayr is closer, but not greawt either.



\subsection{Structural insights}

The flexibility of SMIRNOFF99Frosst is similiar to that of GAFF 1.7.
The flexibility of GAFF v2.1 is similar to other carbohydrate specific force fields.

Include the torsion profile plots, too.

\section{Conclusions and future work}

%%%%%%%%%%%%%%%%%%%%%%%%%%%%%%%%%%%%%%%%%%%%%%%%%%%%%%%%%%%%%%%%%%%%%%%%%%%%%%%%%%%%%%%%%%%%%%%%%%%%%%
% Code and Data Availability
%%%%%%%%%%%%%%%%%%%%%%%%%%%%%%%%%%%%%%%%%%%%%%%%%%%%%%%%%%%%%%%%%%%%%%%%%%%%%%%%%%%%%%%%%%%%%%%%%%%%%%

\section{Code and data availability}
\begin{itemize}
	\item Code to convert from GAFF to SMIRNOFF99Frosst
	\item Code to setup the simulations using pAPRika
	\item Code to analyze the simulation results
	\item OpenFF code and link to SMIRNOFF99Frosst
\end{itemize}
%%%%%%%%%%%%%%%%%%%%%%%%%%%%%%%%%%%%%%%%%%%%%%%%%%%%%%%%%%%%%%%%%%%%%%%%%%%%%%%%%%%%%%%%%%%%%%%%%%%%%%
% Author Contributions 
%%%%%%%%%%%%%%%%%%%%%%%%%%%%%%%%%%%%%%%%%%%%%%%%%%%%%%%%%%%%%%%%%%%%%%%%%%%%%%%%%%%%%%%%%%%%%%%%%%%%%%
\section{Author Contributions}
Conceptualization, DRS, NMH, JDC, MKG; Methodology, DRS, NMH; Software, DRS, NMH; Formal Analysis, DRS, NMH, JDC, MKG; Investigation, DRS, NMH; Resources, MKG, JDC;  Data Curation, DRS, NMH; Writing-Original Draft, DRS, NMH; Writing - Review and Editing, DRS, NMH, JDC, MKG; Visualization, DRS; Supervision, JDC, MKG; Project Administration, MKG; Funding Acquisition, MKG.

%%%%%%%%%%%%%%%%%%%%%%%%%%%%%%%%%%%%%%%%%%%%%%%%%%%%%%%%%%%%%%%%%%%%%%%%%%%%%%%%%%%%%%%%%%%%%%%%%%%%%%
% Acknowledgments 
%%%%%%%%%%%%%%%%%%%%%%%%%%%%%%%%%%%%%%%%%%%%%%%%%%%%%%%%%%%%%%%%%%%%%%%%%%%%%%%%%%%%%%%%%%%%%%%%%%%%%%
\section{Acknowledgments}
%%%%%%%%%%%%%%%%%%%%%%%%%%%%%%%%%%%%%%%%%%%%%%%%%%%%%%%%%%%%%%%%%%%%%%%%%%%%%%%%%%%%%%%%%%%%%%%%%%%%%%
% Disclosures 
%%%%%%%%%%%%%%%%%%%%%%%%%%%%%%%%%%%%%%%%%%%%%%%%%%%%%%%%%%%%%%%%%%%%%%%%%%%%%%%%%%%%%%%%%%%%%%%%%%%%%%
\section{Disclosures}
The authors declare the following competing financial interest(s): MKG has an equity interest in and is a cofounder and scientific advisor of VeraChem LLC.

\bibliography{references}

%%%%%%%%%%%%%%%%%%%%%%%%%%%%%%%%%%%%%%%%%%%%%%%%%%%%%%%%%%%%
%%% APPENDICES
%%%%%%%%%%%%%%%%%%%%%%%%%%%%%%%%%%%%%%%%%%%%%%%%%%%%%%%%%%%%

\appendix
\begin{figure}[tb]
\centering
\includegraphics[width=0.49\textwidth]{images/atom-names-trimer.png}
\includegraphics[width=0.49\textwidth]{images/gaff-atom-types.png}
\caption{Atom names (left) and GAFF atom types (right) for a glucose monomer in alpha-cyclodextrin, shown with two flanking monomers. The remaining three glucose monomers are hidden for clarity.}
\label{fig:atom-names}
\end{figure}

\section{List of abbreviations}
APR, attach-pull-release; CD, cyclodextrin; GAFF, Generalized AMBER Force Field
\end{document}
